\documentclass[12pt, a4paper, oneside]{article}

\usepackage{setspace}
\usepackage{xeCJK}
\usepackage{geometry}
\usepackage[colorlinks,linkcolor=blue]{hyperref}

\geometry{left=2cm} 
\geometry{bottom=2cm} 
\geometry{top=2cm} 
\geometry{hcentering}

\setmainfont{Times New Roman}
\setCJKmainfont[BoldFont=SimHei,ItalicFont=KaiTi]{SimSun}
\renewcommand{\baselinestretch}{1.5}

\title{Review on \textit{Rapid Forward Modelling of Logging-While-Drilling Neutron-Gamma Density Measurements}}
\author{余一凡}
\date{\today}

\begin{document}

    \maketitle
    \section{Background}
    I choose the paper named \href{https://library.seg.org/doi/10.1190/geo2018-0142.1}{\textit{"Rapid Forward Modelling of Logging-While-Drilling Neutron-Gamma Density Measurements"}}, written by \textit{Mathilde Luycx and Carlos Torres-Verdin} \footnote{The University of Texas, Austin}.
    The article is published on \href{https://seg.org/Publications/Journals/Geophysics}{\textit{Geophysics}} (Volume 83, Issue 6), the publisher of which is \textit{Society of Exploration Geophysicists}, in 2018.
    My previous translate part of homework is related with Gamma-ray emitting from rocks, it's mainly used in exploration geophysics. Therefore, I choose this paper which focuses on advancing the speed of gamma-ray logging technology.
    \section{Main Techiniques and Results}

    \section{What I learned}

    \section{Source Reliability}

    \section{Conclusions Validity}
    \end{document}