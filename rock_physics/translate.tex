\documentclass[a4paper,12pt,onecolumn]{article}
\usepackage{changepage}
\usepackage{amsmath,amsfonts,amssymb}
\usepackage{graphicx}
\usepackage{mathptmx}
\usepackage{booktabs}
\usepackage[labelfont=bf]{caption}
\usepackage{indentfirst}
\usepackage{caption}
\usepackage{enumitem}
\usepackage{subfigure}
\usepackage{fontspec}
\usepackage{xeCJK}
\usepackage{float}
\usepackage{multicol}
\usepackage{geometry}
\usepackage[version=4]{mhchem}
\usepackage{multirow}
\usepackage{longtable}


% Please change the following fonts if they are not available.
\setmainfont{Times New Roman}
\setCJKmainfont[BoldFont=SimHei,ItalicFont=KaiTi]{SimSun}

\addtolength{\topmargin}{-54pt}
%\setlength{\textwidth}{18cm}
%\setlength{\textheight}{30cm}
\geometry{left=2cm} 
\geometry{bottom=2cm} 
\geometry{top=2cm} 
\geometry{hcentering} 

\renewcommand{\baselinestretch}{1.1}
\parindent 2em

\begin{document}

\setlength{\textwidth}{15.50cm}
\renewcommand\tablename{表}
\renewcommand\figurename{图}

\vspace{1cm}
\textbf{\huge 第五章}

\vspace{1.5cm}
\textbf{\Huge 放射性性质}
\setcounter{section}{5}

\renewcommand {\contentsname}{本章大纲}
%
\vspace{1.5cm}
\begin{center}
\parbox{\textwidth}{
    \begin{adjustwidth}{0.2cm}{0.2cm}
    \begin{multicols*}{2}
    \tableofcontents
    \end{multicols*}
    \end{adjustwidth}
}
\end{center}

\renewcommand {\baselinestretch}{1.1}
\renewcommand {\thetable} {\thesection{}.\arabic{table}}
\renewcommand {\thefigure} {\thesection{}.\arabic{figure}}
\renewcommand {\theequation} {\thesection{}.\arabic{equation}}
\setcounter{subsection}{0}
\subsection{\large 基本概念}

放射性通常与物质的原子结构有关:一个原子由一个原子核和一些环绕着核的电子$e^-$组成。电子通常占据不同层级(比如K,L,M层级)。电子核是由电中性的中子$n$和带正电的质子$p^+$组成。
表\ref{tb1}给出了这些成分的基本性质。

\begin{table}[h]
    \centering
    \caption{原子性质}
    \label{tb1}
    \setlength{\tabcolsep}{15mm}
    \begin{tabular}{c|c|c}
        \hline
        \quad &质量(Kg)&电荷量\\
        \hline
        质子$p^n$&$1.67*10^{-27}$&$+1.602*10^{-19}$C\\
        \hline
        中子$n$&$1.67*10^{-27}$&电中性,$\pm 0$\\
        \hline
        电子$e^-$&$9.11*10^{-31}$&$-1.602*10^{-19}$C\\
        \hline
    \end{tabular}
\end{table}

原子的质子数是原子的序数$Z$。一个电中性,原子序数为$Z$的原子有$Z$个环绕电子。原子的质子数和中子数的和是原子的质量数$A$。

某种元素通常表示成$_{Z}\mbox{元素符号}^A$。比如,氦元素可以表示为$_{2}He^4$。

每一个元素有一系列独特的离散的能级。一个稳定的原子核处于最低的能级(基态)。位于更高的能级时,它处于激发态,不稳定。在降低至最小能级的过程中,原子核会释放能量。
这种衰变过程所需的时间从几分之一秒到数百万年不等,具体取决于元素种类。衰变过程遵循等式:
\begin{equation}\label{eq1}
    N(t)=N_0 \cdot \exp(-C_d \cot t)=N_0 \cdot \exp(-0.693 \cdot \frac{t}{t_{1/2}})
\end{equation}

$N_0$表示开始计算时原子核数目,$N(t)$表示t时间后原子核数目,$C_d$表示该种元素衰减系数,$t_{1/2}$表示该种元素半衰期。

在统计学中,原子核衰变过程可以用泊松分布来表示。对于泊松分布,标准差$\sigma$与观测到的核数$n$有关
\begin{equation}\label{eq2}
    \sigma = \sqrt{n}
\end{equation}

对于实际测量,这会导致不确定度由测量得到的核数控制;所以,这依赖于辐射的强度,检测的灵敏度和观测时间。因此,这限制了放射性测量工具记录的速度,必须快速记录。

\subsection{天然放射性}

有三种天然放射性粒子流:
\begin{enumerate}
    \item 阿尔法射线(带正电的粒子辐射)
    \item 贝塔射线(带负电的粒子辐射)
    \item 伽马射线 (高能电磁波,光子)
\end{enumerate}

阿尔法和贝塔射线的穿透深度都很小。所以在实际应用中,通常采用伽马射线。

伽马射线的能量通常写成电压($eV$)形式,或者频率($Hz$)形式:
\begin{equation}\label{eq3}
    E=h \cdot f = 4.1357*10^{-15} \cdot f
\end{equation}

$h$是普朗克常数$h=4.1357*10^{-15}eVs=6.6261*10^{-34}Js$。

\subsubsection{岩石中天然伽马射线的起源}

元素通常有多种同位素形式。如果这些同位素不稳定,它们就会衰变到稳定形式,并且放出辐射。天然岩石中只有三种放射性衰减过程,产生可以测量的伽马射线:
\begin{itemize}
    \item 铀元素(半衰期$4.4*10^9$年)
    \item 钍元素(半衰期$1.4*10^9$年)
    \item 钾元素$K^40$(半衰期$1.3*10^9$年)
\end{itemize}

铀元素$U-238$($_{92}U^{238}$)和钍元素$Th-232$($_{90}Th^{232}$)经过一系列的衰变,最终分别得到稳定的铅元素$Pb-206$($_{82}Pb^{206}$)和$Pb-208$($_{82}Pb^{208}$)。衰变得过程导致辐射光谱中出现离散的能级。铀元素和钍元素的光谱中都有特征性的尖峰:铀元素为1.76MeV,钍元素为2.62MeV。

$0.0119\%$钾元素是放射性同位素$_{19}K^{40}$。衰变过程释放能量为1.46MeV的伽马射线。衰变的最终产物是氩元素($_{18}Ar^{40}$)图\ref{fig1}
\begin{figure}[h]
    \centering
    \includegraphics[width=14cm]{figures/fig1.jpg}
    \caption{铀元素,钍元素,钾元素衰变释放的伽马射线中的能量。垂直线段的长度与辐射强度成比例\textit{Schlumberger(1989a,b)}}
    \label{fig1}
\end{figure}

\subsubsection{矿物中的钾,铀,钍}

钾元素广泛地以化学方式结合在许多矿物中;有三类主要的含钾矿物:
\begin{enumerate}
    \item 黏土矿物,钾元素通常或者存在于矿物结构中,或者被黏土颗粒吸收;一个实例是伊利石(\ce{(K,H3O)Al2(Si3Al)O10(H2O,OH)2})。不同黏土矿物中地钾元素含量也不同:伊利石钾含量高($3.5-8.3\%$),相反高岭石钾含量比较低。
    \item 成岩矿物,钾元素以化学方式存在于矿物结构中;经典的例子是钾长石(正长石\ce{KAlSi3O8},微斜长石\ce{KAlSi3O8})和云母。
    \item 盐结晶,钾元素通常保存为盐(钾盐\ce{KCl},光卤石\ce{KCl}\ce{MgCl2(H2O)6})。
\end{enumerate}

钾元素被看作是“起源于侵蚀的元素,它常常被剥蚀和长距离搬运”(\textit{Baker Atlas, 1985})

钍元素通常与酸性或中性岩石关联。它非常的稳定并且不会被溶解。由于风化和搬运作用,钍元素与碎屑沉积物一起沉积,从不沉积在纯化学物质沉积物中(碳酸盐、文石)。因此,在碳酸盐岩储层中,钍元素是非常重要的黏土指示物。由于被粘土矿物吸附,它和其他富集的重矿物一起沉积在海洋中。所以,钍通常被认为是一种海洋元素(\textit{Baker Atlas, 1985})

铀元素也通常与酸性或中性岩石关联——通常浓度为4.65ppm。由于风化和搬运,不像钾元素和钍元素,铀元素可以被溶解,会被搬运到河流或者海洋中。铀盐不稳定,会进入沉积物中。铀元素进入沉积物中有三种途径(\textit{Serra 1984,2004})
\begin{enumerate}
    \item 在酸性还原环境下化学沉降(pH 2.5-4.0)
    \item 被植物或动物吸收
    \item 与磷反应
\end{enumerate}

酸还原条件尤其存在于停滞、缺氧的水中。沉降率相对较低,通常会产生黑色页岩。(\textit{Rider, 1986})

受搬运、沉积过程以及化学环境控制,铀存在于碎屑和化学沉积物(页岩、砂岩、砾岩、碳酸盐岩),也常见于凝灰岩和磷酸盐中。在还原作用下形成的富含有机物的碳酸盐岩中,铀的含量通常非常高。这些放射性的碳酸盐岩是丰富的储层。铀也被粘土矿物吸附;页岩中铀含量过高表明页岩为烃源岩。(\textit{Baker Atlas, 1985})

\textit{Rider, 1996}指出“铀元素作为一种独立的成分,它没有像钾一样结合于岩石的化学结构,反而松散的与第二成分结合。因此,它在沉积物中的分布非常不均匀。

三种放射性成分的主要发生情况如下(\textit{Schilumberger, 1982}):
\begin{enumerate}
    \item 钾:云母,钾长石,黏土,放射性盐岩
    \item 钍:页岩,重矿物
    \item 铀:磷矿,有机质
\end{enumerate}

表\ref{tb2}给出矿物的钾、铀和钍含量。一些重要的造岩矿物如石英,方解石,白云石和硬石膏没有列出。因为作为矿物,它们与钾、铀和钍无关。
\begin{center}
	\begin{longtable}{lllllp{3cm}p{2cm}p{2cm}p{2cm}p{1.5cm}}
    \caption{一些矿物的钾、铀和钍含量;[]中的值表明平均值}\\
    \label{tb2}\\
        \hline
        矿物&钾(\%)&铀(ppm)&钍(ppm)&来源\\
        \multirow{2}*{斜长石}&0.54&0.02-5.0&0.01-3.0&BA\\
        ~&0.54&0.2-5.0&0.5-3.0&Sch\\
        \hline
        \multirow{2}*{正长石}&11.8-14.0&0.2-3.0&0.01-7.0&BA\\
        ~&$11.8^a$(14理想状态)&0.2-3.0&3-7&Sch\\
        \hline
        \multirow{2}*{微斜长石}&10.9&0.2-3.0&0.01-7.0&BA\\
        ~&$10.9^a$(16理想状态)&\quad &\quad &Sch\\
        \hline
        \multirow{3}*{黑云母}&6.2-10&1-40&0.5-50&Hu\\
        ~&6.2-10.1[8.5]&1-40&0.5-50&Sch\\
        ~&6.7-8.3&\quad &$<0.01$&BA\\
        \hline
        \multirow{3}*{白云母}&7.8-9.8&\multirow{3}*{2-8}&0-25&Hu\\
        ~&7.8-9.8&~&$<0.01$&BA\\
        ~&7.9(9.8理想状态)&~&10-$25^b$&Sch\\
        \hline
        \multirow{3}*{伊利石}&3.5-8.3&1-5&10-25&Hu\\
        ~&4.5&1.5&$<2.0$&BA\\
        ~&3.5-8.3[6.1]&1.5 &10-25&Sch\\
        \hline
        \multirow{3}*{高岭石}&0-0.6&1-12&6-47&Hu\\
        ~&0.42&1.5-3&6-19&BA\\
        ~&0-$0.6^c$[0.35]&1.5-9 &6-42&Sch\\
        \hline
        \multirow{2}*{亚氯酸盐}&0-0.3&\quad &3-5&Hu\\
        ~&0-0.35[0.1]&\quad &\quad &Sch\\
        \hline
        \multirow{2}*{蒙皂石}&0-1.5&1-21&6-44&Hu\\
        ~&0-0.6[0.22]&\quad &10-24&Ri\\
        \hline
        \multirow{2}*{蒙脱石}&0.16&2-5&14-24&BA\\
        ~&0-$4.9^d$[1.6]&2-5&10-24&Sch\\
        \hline
        \multirow{2}*{铝土矿}&\quad &3-30[8.0]&8-132[42]&BA\\
        ~&\quad &3-30&10-132&Sch\\
        \hline
        \multirow{2}*{膨润土}&$<0.5$&1-21[5.0]&6-50&BA\\
        ~&\quad &1-36&4-55&Sch\\
        \hline
        \multirow{3}*{海绿石}&5.08-5.30&\quad &\quad &BA\\
        ~&3.2-5.8[4.5]&\quad &$<10$&Sch\\
        ~&3.2-5.8[4.5]&\quad &2-8&Ri\\
        \hline
        磷酸盐&\quad&1000-350&1-5&BA\\
        锆石&\quad &300-3000&100-2500&BA\\
        榍石&\quad &100-700&100-600&BA\\
        绿帘石&\quad &20-50 &50-500&BA\\
        磷灰石&\quad &5-150&20-150&BA\\
        独居石&\quad &500-3000&$(2.5-20)*10^4$&BA\\
        钾盐&52.4&\quad &\quad&Sch\\
        菱镁矿&18.8&\quad &\quad&Sch\\
        凯因石&15.7&\quad &\quad&Sch\\
        光卤石&14.1&\quad &\quad&Sch\\
        \hline
\end{longtable}
\parbox{15cm}{表中:BA指(\textit{Baker Atlas, 1985}),Ri指(\textit{Rider, 1986}),Sch指(\textit{Schlumberger, 1983}),Hu指(\textit{Hurst, 1990})\\
\textit{a:指初始搬运
b:没有钍的纯白云母。然而,在沉积岩中,白云母的沉积(或云母)通常伴随着更细的重矿物沉积,重矿物承载铀和钍。
c:由于长石降解不完全,高岭石有时含有更多的 K。 自生高岭石不含 K 和 Th
d:一些蒙脱石可能对应于不完全降解的白云母或不完整的通过成岩作用转化为伊利石}}
\end{center}

\begin{table}[H]
    \centering
    \caption{铀、钍和钾在地壳中的平均含量值。来自\textbf{Heier and Rogers(1963)}[H]和\textbf{Prutkina and Saskin(1975)}[P]}
    \label{tb3}
    \begin{tabular}{c|c|c|c|c}
        地壳类别&钾(\%)&铀(ppm)&钍(ppm)&来源\\
        \hline
        地壳&2.5&2.5&13&P\\
        地壳&2.1&2.1&7.8&H\\
        大洋地壳&0.87&0.64&2.8&H\\
        大陆地壳&2.6&2.8&10&H\\
        \hline
    \end{tabular}
\end{table}

\subsubsection{岩石中的钾、铀和钍含量}

岩石的天然放射性是由于三种放射性元素:铀、钍和钾的分布。铀、钍和钾在不同地壳岩石中的浓度变化很大,这是因为不同岩石的矿物组成,成因和环境不同。地壳的平均含量在表\ref{tb3}中展示。一些挑选过的值在表\ref{tb4}中。更多数据在(\textit{Hacck 1982 and Clark 1966})中可以被找到。

表\ref{tb4}展示了不同岩石的铀、钍和钾含量。这与岩石对应的造岩矿物中元素含量对应(表\ref{tb2})。

\begin{center}
	\begin{longtable}{lllllp{3cm}p{2cm}p{2cm}p{2cm}p{1.5cm}}
		\caption{一些岩石中的钾,铀和钍的含量}\\
		\label{tb4}\\
		\hline\noalign{\smallskip}
		岩石种类 & 钾(\%) & 铀(ppm) & 钍(ppm) & 来源\\
		\hline
        \multicolumn{5}{l}{侵入岩}\\
        \hline
		花岗岩 & 2.75-4.26 & 3.6-4.7 & 19-20 & BA \\
        花岗质岩石(平均) & 4.11 & 4.35 & 15.2 & Sch \\
        花岗质岩石 & 2.3-4.0 & 2.1-7.0 & 8.3-40 & Do \\
        黑云母花岗岩 & 3.4 & 4.0 & 15 & D \\
        \multirow{2}*{正长岩} & 2.7-4.9 & 2500 & 1300 & BA \\
        ~ & 2.63 & 2500 & 1338 & Sch \\
        辉长岩 & 0.46-0.58 & 0.84-0.90 & 2.70-2.85 & Sch \\
        \multirow{2}*{花岗闪长岩} & 2-2.5 & 2.6 & 9.3-11 & BA, Sch \\
        ~ & 2.3 & 2.1 & 8.3 & Do \\
        \multirow{2}*{闪长岩} & 1.1 & 2.0 & 8.5 & BA, Sch \\
        ~ & 1.8 & 1.8 & 6.0 & Do \\
        纯橄榄岩 & $<0.02$ & $<0.01$ & $<0.01$ & BA, ch \\
        纯橄榄岩 辉石岩 & 0.15 & 0.03 & 0.08 & D \\
        橄榄岩 & 0.2 & 0.01 & 0.05 & BA, Sch \\
        \hline
        \multicolumn{5}{l}{喷出岩}\\
        \hline
        \multirow{2}*{流纹岩} & 4.2 & 5.0 & \quad & BA \\
        ~ & 2-4 & 2.5-5 & 6-15 & Sch \\
        粗面岩 & 5.7 & 2-7 & 9-25 & Sch \\
        碱性玄武岩 & 0.61 & 0.99 & 4.6 & BA, Sch \\
        高地玄武岩 & 0.61 & 0.53 & 1.96 & Sch  \\
        \multirow{2}*{安山岩} & 1.7 & 0.8 & 1.9 & Sch \\
        ~ & 1.7 & 1.2 & 4.0 & D \\
        英安岩 & 2.3 & 2.5 & 10.0 & D  \\
        英安流纹岩 & 3.7 & 4.7 & 19 & Do \\
        \hline
        \multicolumn{5}{l}{变质岩}\\
        \hline
        片麻岩-瑞士阿尔卑斯山 & 0.32-4.7[3.11] & 0.9-24[4.95] & 1.2-25.7[13.1] & RyCe \\
        片麻岩(KTB,德国) & $2.28\pm 0.17$ & $2.6\pm 1.2$ & $8.2\pm 2.0$ & B \\
        榴辉岩 & 0.8 & 0.2 & 0.4 & D \\
        角闪石-瑞士阿尔卑斯山 & 0.11-2.22[1.23] & 0-7.8[1.65] & 0.01-13.7[3.0] & RyCe \\
        角闪石 & 0.6 & 0.7 & 1.8 & Do \\
        准基性岩(KTB,德国) & $0.6\pm 0.5$ & $2.5\pm 1.6$ & $2.5\pm 1.6$ & B \\
        片岩-瑞士阿尔卑斯山 & 0.39-4.44[2.23] & 0.4-3.7[2.14] & 1.6-17.2[9.73] & RyCe \\
        \multirow{2}*{石英岩} & 0.6 & 0.8 & 3.1 & D \\
        ~ & 0.9 & 0.6 & 1.8 & R \\
        大理岩 & 0.2 & 1.1 & 2.2 & D \\
        \hline
        \multicolumn{5}{l}{沉积岩}\\
        \hline
        碳酸盐岩 & 0.0-2.0[0.3] & 2.8-2.5[2.2] & 0.1-7.0[1.7] & BA \\
        \multirow{2}*{石灰岩} & 0.3 & 1.6 & 1.8 & D \\
        ~ & 0.3 & 2.0 & 1.5 & R \\
        \multirow{2}*{白云岩} & 0.4 & 3.7 & 2.8 & D \\
        ~ & 0.7 & 1.0 & 0.8 & R \\
        泥灰岩 & 0.8 & 2.8 & 2.5 & D \\
        硬石膏 & 0.4 & 0.1 & 0.3 & R \\
        盐岩 & 0.1 & 0.02 & 0.3 & R \\
        砂岩 & 0.7-3.8[1.1] & 0.2-0.6[0.5] & 0.7-2.0[1.7] & BA \\
        硬砂岩 & 1.3 & 2.0 & 7.0 & R \\
        页岩(200个样品) & 2.0 & 6.0 & 12.0 & BA \\
        普通页岩 & 1.6-4.2[2.7] & 2-13 & 3-47 & BA \\
        页岩 & 2.7 & 3.7 & 12 & R \\
        含油页岩(科罗拉多) & $<4.0$ & 最高500 & 1-30 & BA \\
        黑页岩 & 2.6 & 20.2 & 10.9 & R \\
        标准成分北美页岩 & 3.2 & 2.66 & 12.3 & KGS \\
        黏土(北大西洋,加勒比) & 2.5 & 2.1 & 11 & KaHa \\
        黏土(更新世) & 1.9-2.5 & 1.1-3.8 & 5.7-10.2 & \quad \\
        黏土,泥(第三纪) & 1.3-3.1 & 1.2-4.3 & 1.4-9.3 & \quad \\ 
	\end{longtable}

    \parbox{15cm}{BA,\textit{Baker Atlas(1985)}; Sch,\textit{Schlumberger (1982)}; Do, \textit{Dortman (1976)}; RyCe, \textit{Rybach and Cermak (1982)}; B, \textit{Buker et al. (1989)}; KaHa, \textit{Kappelmeyer and Haenel (1974)}; D, \textit{Dobrynin et al. (2004)}; R, \textit{Rybach (1976)}; KGS, \textit{Kanas Geological Survey (2010)}. []中的值表示平均}
\end{center}

\begin{figure}[htb]
    \centering
    \caption{岩石天然放射性大体趋势}
    \label{fig2}
    \includegraphics[width=10cm]{figures/fig2.jpg}
\end{figure}

\begin{table}
    \centering
    \caption{火成岩中铀、钍和钾平均含量。来自(\textit{Dortman (1976), Saporosec (1978), Fertl (1979), Kobranova (1989)})}
    \label{tb5}
    \begin{tabular}{l|l|l|l|l}
        \hline
        岩石种类 & 铀(ppm) & 钍(ppm) & 钾(\%) & 钍/铀 \\
        \hline
        花岗岩 & 4-7 & 15-40 & 3.4-40 & 3.5-5.6 \\
        花岗闪长岩 & 2.1 & 8.3 & 2.3 & 4.0 \\
        闪长岩 & 1.8 & 6.0 & 1.8 & 3.3 \\
        辉长岩,辉绿岩 & 0.6 & 1.8 & 0.7 & 3.0 \\
        橄榄岩 & 0.03 & 0.08 & 0.15 & 2.7 \\
        玄武岩 & 0.7 & 2.3 & 1.0 & 3.2 \\
        \hline
    \end{tabular}
\end{table}

图\ref{fig2}表示了由于钾、铀和钍的分布导致的岩石中天然放射性的大体趋势。
\begin{itemize}
    \item 火成岩的放射性从基性岩到酸性岩递增。
    \item 沉积岩的放射性从砂岩到页岩递增,因为黏土矿物的成分增加。
\end{itemize}

\paragraph{5.2.3.1 火成岩}

通常情况下,这三种元素在酸性岩石中比基性岩更多。有一个例外是碱性长石类岩石。岩浆岩中的高放射性主要与含铀和钍的副矿物的存在有关(\textit{Kobranova, 1989})。在每种岩石类型中,元素含量的分布所产生的放射性是有代表性的。(表\ref{tb5})

\begin{table}[!h]
    \centering
    \caption{一些火成岩的铀、钍和钾含量,源自USGS;\textit{Adams and Gasparini 1970}}
    \label{tb6}
    \begin{tabular}{l|l|l|l}
        \hline
        岩石种类 & 铀(ppm) & 钍(ppm) & 钾(\%)\\
        \hline
        \multicolumn{4}{l}{酸性侵入}\\
        \hline
        \multicolumn{4}{l}{花岗岩}\\
        罗德岛 & 1.32-3.4(1.99) & 21.5-26.6(25.2) & 3.92-4.8(4.51) \\
        罗德岛 & 1.3-4.7(4) & 6.5-80(52) & 5.06-7.4(5.48) \\
        前寒武纪 & 3.2-4.6 & 14-27 & 2-6 \\
        平均 & 4.35 & 15.2 & 4.11 \\
        正长岩 & 2500 & 1338 & 2.63 \\
        \hline
        \multicolumn{4}{l}{基性侵入}\\
        \hline
        辉长岩 & 0.84-0.9 & 2.7-3.85 & 0.46-0.58 \\
        花岗闪长岩 & 2.6 & 9.3-11 & 2-2.5 \\
        闪长岩 & 2 & 8.5 & 1.1 \\
        纯橄岩 & 0.01 & 0.01 & 0.02 \\
        橄榄岩 & 0.01 & 0.05 & 0.2 \\
        \hline
        \multicolumn{4}{l}{酸性喷出}\\
        \hline
        流纹岩 & 2.5-5 & 6-15 & 2-4 \\
        粗面岩 & 2-7 & 9-25 & 5.7 \\
        \hline
        \multicolumn{4}{l}{基性喷出}\\
        \hline
        碱性玄武岩 & 0.99 & 4.6 & 0.61 \\
        高地玄武岩(俄勒冈) & 0.53 & 1.96 & 0.61 \\
        碱性橄榄石 & 1.4 & 3.9 & 1.4 \\
        玄武岩 & 1.2-2.2(1.73) & 5.5-15(6.81) & 1.4-3.23(1.68) \\
    \end{tabular}
\end{table}

为了比较,表\ref{tb6}表明了USGS标准中的放射性值和范围。

\paragraph{5.2.3.2 变质岩}

\begin{figure}[!h]
    \centering
    \caption{0-3900米的间隔的钻孔岩心伽马射线计数率频率分布;n,测量数; 1、变基岩; 2、片麻岩。(\textit{Bucker et al. 1989})}
    \label{fig3}
    \includegraphics[width=15cm]{figures/fig3.jpg}

\end{figure}

变质岩中铀、钍和钾的含量是由离析物(火成岩或沉积岩)的原始含量决定,并且可能已被变质过程改变。元素被吸附,并根据变质程度重新分配(\textit{Rybach and Cermak, 1982})。图\ref{fig3}显示取自KTB的样本测量值以及两个主要岩性单元的分布。

放射性元素的含量经常随着变质作用的增加而减少。\textit{Rybach and Cermak(1982)} 发现,“在麻粒岩相的岩石中,由渐进的变质作用引起的铀和钍的减少最为明显。……由于脱水反应(中地壳水平)或由于地壳底部附近的部分熔化(混合岩),引起铀和钍有向上的地壳迁移趋势……钾似乎或多或少不受这些过程的影响”。变质岩的平均 Th/U 比值可能会偏离典型的侵入岩的值(表\ref{tb7})。 “这是由于钾、钍和铀在变质过程中的不等损失,其中铀的迁移性有很大影响”,根据\textit{Haack (1982)}。

\begin{table}[!h]
    \centering
    \caption{变质岩中铀、钍和钾含量以及Th/U的平均值和范围。 D, \textit{Dortman (1976)}; B, \textit{Bucker (1989)}; R, \textit{Rybach and Cermak (1982)}; P, \textit{Puzankov et al.(1977) and Rybach (1976)}}
    \label{tb7}
    \begin{tabular}{l|l|l|l|l|l}
        \hline
        岩石种类 & 铀(ppm) & 钍(ppm) & 钾(\%) & 钍/铀 & 来源 \\
        \hline
        \multirow{2}*{石英岩-瑞士阿尔卑斯山} & 0.4 & 2.2 & 1.06 & 5.5 & R \\
        ~ & 0.2-0.6 & 2.1-2.3 & 0.95-1.16 & \quad & \quad \\
        片麻岩(KTB,德国) & $2.8\pm 1.2$ & $8.2\pm 2.0$ & $2.28\pm 0.17$ & 3.2 & B \\
        \multirow{2}*{片麻岩-瑞士阿尔卑斯山} & 4.95 & 13.1 & 3.11 & 2.64 & R \\
        ~ & 0.9-24 & 1.2-25.7 & 0.32-4.71 & 0.69-8.2 & \quad \\
        榴辉岩 & 0.2 & 0.4 & 0.8 & 2.0 & D \\
        角闪石 & 0.7 & 1.8 & 0.6 & 2.6 & D \\
        \multirow{2}*{角闪石-瑞士阿尔卑斯山} & 1.65 & 3.0 & 1.23 & 1.82 & R \\
        ~ & 0-7.8 & 0.01-13.7 & 0.11-2.22 & 0.01-3.5 & \quad \\
        准基性岩(KTB,德国) & $0.6\pm 0.5$ & $2.5\pm 1.6$ & $1.05\pm 0.16$ & 4.2 & B \\
        斜长屑混合岩 & 0.8 & 2.3 & 1.7 & 2.9 & D \\
        硅线石-堇青石片麻岩 & 1.3 & 4.2 & 2.6 & 3.2 & D \\
        \multirow{2}*{片岩-瑞士阿尔卑斯山} & 2.14 & 9.73 & 2.23 & 4.6 & R \\
        ~ & 0.4-3.7 & 1.6-17.2 & 0.39-4.44 & 0.25-2.28 & \quad \\
        板岩-勘察加半岛 & 2.0 & 6.4 & 1.83 & 3.2 & P\\
    \end{tabular}
\end{table}

\paragraph{5.2.3.3 沉积岩}

“在沉积岩中,相对平均丰度的可预测性低于在火成岩中。平均而言,钾的有效浓度较低,钍和铀的含量大致相同。作为一类沉积岩,碳酸盐岩的天然放射性最低……一般来说,页岩比其他沉积岩具有更强的天然放射性;因此,伽马射线探仪可用于区分页岩和其他沉积岩”。\textit{Hearst and Nelson, 1985}

黏土成分与沉积岩的放射性强度的相关性对识别储层非常重要:

\begin{itemize}
    \item 分辨黏土层和砂层
    \item 确定黏土成分
    \item 识别黏土类型
\end{itemize}

如果可以进行伽马射线光谱测量,推荐使用钍和钾进行页岩成分推导。\textit{Fertl(1983)} 指出伽马射线光谱记录的钍曲线允许定量测量粘土体积,尽管存在铀和钾的影响。(可能是由于云母,尤其是白云母,例如在北海的侏罗纪砂岩)。

一些关于砂岩的天然放射性讨论实例在\textit{Schulumberger (1982)}:

\hspace{2em}
长石砂岩或弧形砂岩有含钾的长石砂:因此它们显示出非常低的Th/K比率$(<10^{-4}$,同时也有比纯石英($2.65*10^3 kg/m^3$)低的基质密度,因为长石的低密度($2.52-2.53*10^3km/m^3$)。

\hspace{2em}
云母砂岩:云母含钾;因此,钾云母砂岩的钾含量高于纯砂岩。由于含含钍矿物,钍含量也较高。因此,Th/K比率接近于$2.5*10^{-4}$。云母砂岩的密度高于纯石英的密度,因为云母密度较高($2.8-3.1*10^3kg/m^3$)。

\hspace{2em}
砂岩中的重矿物:重矿物如锆石、花岗岩、独居石等是含钍和铀的。因此,在这些砂岩中,只有钍和铀含量高;钾水平非常低。因此,这种类型的砂岩显示出非常高的Th/K比率。

\begin{figure}[!h]
    \centering
    \includegraphics[width=15cm]{figures/fig4.jpg}
    \caption{三种典型的伽马射线光谱测量;来自大洋钻探项目181和189}
    \label{fig4}
\end{figure}

参考粘土矿物和页岩中存在的钍、铀、钾的发生机理,\textit{Rider (1996)}说:“总结:指示页岩的存在,大多数情况下可以使用钍,部分情况下可以使用钾,但根本不应该使用铀。这显然对伽马射线记录的使用有影响:它不一定是合适的页岩探测方法”。

\begin{figure}[!h]
    \centering
    \includegraphics[width=10cm]{figures/fig5.jpg}
    \caption{碳酸盐岩中的伽马射线}
    \label{fig5}
\end{figure}

图\ref{fig4}显示了三种典型的伽马射线光谱测量。

在碳酸盐岩系列中,总的伽马强度是较差的粘土指标,因为测量值与粘土含量无关,而是与铀的存在有关。典型案例是图\ref{fig5}:
\begin{itemize}
    \item 无放射性碳酸盐(化学来源),具有钍和钾含量接近于零。如果铀含量也为零,则该碳酸盐在氧化环境中沉淀。
    \item 如果铀含量变化,则碳酸盐可能是沉积在还原环境中,或对应于碳酸盐中的柱石(其中含有杂质,如铀、有机物甚至粘土矿物),或含磷酸盐。
    \item 如果钍和钾与铀一起存在,这表明碳酸盐中存在粘土(粘土碳酸盐到泥灰岩)。
    \item 如果钾存在或不存在铀,它可以对应于藻类来源的碳酸盐岩或海绿石碳酸盐岩。
\end{itemize}

\subsubsection{光谱和整体测量-API单位}

在地球物理现场实际测量中,测量伽马辐射的两种技术是:光谱测量和整体测量。

\begin{figure}[!h]
    \centering
    \includegraphics[width=15cm]{figures/fig6.jpg}
    \caption{页岩区中钍、钾、铀成分的伽马射线光谱}
    \label{fig6}
\end{figure}

图\ref{fig6}展示了一个全光谱测量页岩区的例子。

对于大部分应用场景,光谱测量技术是通过数据简化实现的:整个光谱分为三个窗口:

\begin{enumerate}
    \item 第一个窗口 1.3-1.6MeV 含有K能量峰(1.46MeV)
    \item 第二个窗口 1.6-2.4MeV 含有U能量峰(1.76MeV)
    \item 第三个窗口 2.4-2.8MeV 含有Th能量峰(2.61MeV)
\end{enumerate}

处理过程,所谓的“光谱分离”(比如\textit{Ellis, 1987; Hearst and Nelson, 1985}),解决了三种元素混合的矩阵算法问题。

在许多情况下,会应用“整体测量技术”:在一个固定的能量水平,测量所有能量。因此,整体测量是三个元素贡献的总和。

\begin{equation}
    \label{eq4}\\
    I_{API}=k\cdot (a\cdot K+U+b\cdot Th)
\end{equation}

U和Th单位是ppm,K是\%。k是仪器的常数,a是与1\%K产生相同辐射的U的浓度值(ppm),b是与1ppmTh产生相同辐射的U的浓度值(ppm)(\textit{Hearst and Nelson, 1985})。

伽马辐射测量的单位由美国石油协会(API)确立。该标准允许在不同的伽马射线计数设备之间进行比较。API单位来自美国休斯顿大学的试验坑。美国石油协会(API)单位标准设施由混凝土建造,掺有镭以提供铀衰变系列,独居石矿石作为钍的来源和云母作为钾的来源。该设施有4.07\%K、24.2ppm Th 和 13.1 ppm U(\textit{Ellis,1987})。该标准设施测量值为200个API,等于页岩平均值的两倍。表\ref{tb8} 给出了一些造岩矿物的平均 API 值。

\begin{table}[!h]
    \centering
    \caption{伽马辐射平均API值}
    \label{tb8}
    \begin{tabular}{l|lp{15cm}p{6cm}}
        \hline
        物质 & API \\
        \hline
        石英,方解石,白云石 & 0 \\
        斜长石(钠长石,钙长石) & 0 \\
        碱性长石 (正长石,微斜长石) & $\approx 220$ \\
        白云母 & $\approx 270$ \\
        黑云母 & $\approx 275 $ \\
        高岭石 & $80-130$ \\
        伊利石 & $250-300$ \\
        亚氯酸盐 & $180-250$ \\
        蒙脱石 & 150-200 \\
        钾盐 & 500+ \\
        光卤石 & $\approx 220$ \\
    \end{tabular}
\end{table}

\subsubsection{应用}

\paragraph{5.2.5.1 岩石剖面}

在火成岩中,总的趋势是辐射强度从超基性到酸性岩石逐渐增加。 这归因于较高的铀、钍含量和云母、碱长石中的钾含量。岩石成分的改变可以改变放射性。

在沉积岩中,无放射性的碳酸盐和砂岩通常显示最低值。放射性随着页岩含量的增加而增加。最高值来自黑色海相页岩。但有一些重要的具体的高辐射情况:
\begin{itemize}
    \item 长石、云母、海绿石(绿砂)含量高的砂岩。
    \item 还原环境中的碳酸盐岩、柱石、磷酸盐。
\end{itemize}

伽马辐射记录允许进行(定性)岩性分析。无放射性(砂岩、碳酸盐岩)和页岩部分可以被分开。在地质应用中,伽马辐射记录是一个工具可以面向:
\begin{itemize}
    \item 通过典型的河道曲线形状进行沉积学研究,上部粗糙、上部分选好等形态
    \item 沉积区井间对比及趋势推导
\end{itemize}

\paragraph{5.2.5.2 页岩成分估计}

\textit{利用天然伽马辐射光谱数据是最好的确定黏土体积的方法... \\ \rightline{Fertl(1983)}}

页岩含量估计值可以从基于伽马测量的页岩含量与放射性同位素含量之间的相关性得到。假设只有页岩或粘土产生辐射;不存在其他“放射性矿物质”。

对于这种应用,在许多情况下使用整体测量。为了消除铀含量变化的影响,建议测量钾和钍的光谱能量贡献。

分析包含两步:

第一步:

计算伽马辐射系数$I_{GR}$:这是这是实际的伽马读数通过无放射性岩石的值(最小伽马读数)和页岩(最大伽马读数)归一化:

\begin{equation}
    \label{eq5}\\
    I_{GR}=\frac{GR-GR_{cn}}{GR_{sh}-GR_{cn}}
\end{equation}

$GR_{cn}$是没有页岩的记录,$GR_{sh}$是页岩区的记录,$GR$是观测区的记录。

这个归一化对于无放射性岩石$I_{GR}=0$,对于页岩(100\%页岩)$I_{GR}=1$。

第二步:

将伽马辐射系数转换为页岩含量:对于这一转换,推荐使用经验公式。一下是经验公式,图\ref{fig4}给了一个例子。线性相关在所有情况下给出最高的页岩含量图\ref{fig7}。

\begin{align}
    V_{sh}=I_{GR}\quad Linear\quad relationship(upper\quad limit)\label{eq6}\\
    V_{sh} = 0.083\cdot (2^{3.7\cdot I_{GR}}-1) \quad Tertiary\quad clastics\label{eq7}\\
    V_{sh} = 0.33\cdot (2^{2.0\cdot I_{GR}}-1)\quad Mesozoic \quad and \quad older \quad rocks \label{eq8}
\end{align}

\begin{figure}[!h]
    \centering
    \includegraphics[width=15cm]{figures/fig7.jpg}
    \caption{伽马辐射系数$I_{GR}$和页岩含量$V_{sh}$的关系。作为例子$I_{GR}=0.35$(打点的线),对于线性关系有$V_{sh}=0.35$。$V_{sh}=0.21$对于第三纪碎屑,$V_{sh}=0.12$对于中生代或更老的碎屑。}
    \label{fig7}
\end{figure}

\paragraph{5.2.5.3 黏土矿物种类}

\begin{figure}[!h]
    \centering
    \includegraphics[width=15cm]{figures/fig8.jpg}
    \caption{K-Th图从光谱记录估计黏土矿物(\textit{Mohammadlou et al., 2010})。上图:Th-K图,下图:在两个区域中识别出的黏土种类的光谱观测数据的Th-K图}
    \label{fig8}
\end{figure}

钍和钾是“重要的矿物质”放射性成分。粘土矿物(以及云母和长石)的特征在于不同比例的钍与钾。图\ref{fig8}中的两种元素或不同位置。这可用于估计主要粘土地层中的矿物,也可用于检测云母或长石。

图\ref{fig8}(下图)展示了一个例子(古生代碳酸盐岩与巴伦支海的碳酸盐岩和硅质碎屑岩混合)。1区显示混合层粘土或伊利石;5区显示出海绿岩或长石岩,砂岩与云母和伊利石。

\textit{Hurst (1990)}发表了批判Th-K作图识别砂岩的黏土种类的分析。主要原因是:

\begin{itemize}
    \item 高岭石和亚氯酸盐岩可忽略的钾含量
    \item 钍形成粉砂大小的离散自生矿物,而不是吸附在粘土矿物表面的趋势
    \item 不足的数据库和统计分析基础
\end{itemize}

\paragraph{5.2.5.4 源岩识别}

对于源岩的研究,铀在不同环境下的行为尤为重要。(\textit{Fertl, 1979})利用Th/U 比率作为衡量标准:

\begin{itemize}
    \item Th/U>7 $\rightarrow$ 陆相,氧化环境
    \item Th/U<7 $\rightarrow$ 海相,灰-绿页岩
    \item Th/U<2 $\rightarrow$ 碱性,黑色页岩,磷酸岩
\end{itemize}

在停滞、缺氧的水中会发生极高的吸附作用(通常产生黑色页岩),沉积物沉积率低(例如北海侏罗纪热页岩)。

\begin{figure}[!ht]
    \centering
    \includegraphics[width=15cm]{figures/fig9.jpg}
    \caption{Th/U值和泥盆纪黑色页岩(弗吉尼亚和肯塔基)生物碳含量关系。数据来源:\textit{Fertl (1983)}}
    \label{fig9}
\end{figure}

图\ref{fig9}展示了泥盆纪黑色页岩中Th/U值和有机碳含量的强相关性。

油气储层中观测到的天然伽马辐射强度变化与原油中铀的含量有关表\ref{tb9}还有在生产过程中的成分改变(\textit{Fertl 1983})。异常高的伽马辐射值在水油接触和采完油且注满水的地层中检测到(\textit{Doering and Smith, 1974; Fertl, 1983; Khusnullin, 1973; Kin and Bradley, 1977; Lehnert and Just, 1979})。

\begin{table}[!h]
    \centering
    \caption{部分原油中铀含量\quad \textit{Fertl (1983)}}
    \label{tb9}
    \begin{tabular}{l|l|l|l}
        \hline
        位置 & 铀($10^{-3}ppm$) & 位置 & 铀($10^{-3}ppm$)\\
        \hline
        阿肯色 & 0.5-2.5 & 俄克拉荷马 & 0.32-1.98 \\
        科罗拉多 & 0.17-0.7 & 堪萨斯 & 0.28-2.6 \\
        蒙塔纳 & 0.12 & 怀俄明 & 0.24-13.5 \\
        新墨西哥 & 0.54 & 利比亚 & 15.0 \\
        
    \end{tabular}
\end{table}

\textit{Toulhoat et al. (1989)}在低渗透性地层(含水层)中进行抽水实验,研究了原位研究离子的保留行为,并发现由于泵压过程,铀含量明显减少。

\subsubsection{衰变放热}

在地壳中,天然放射性元素(钾、铀和钍)衰变产生热量,是构成地热流的一大部分。 地球表面的平均热流约为$65 mW/m^2$,来自大陆地区地幔的热流约为$20 mW/m^2$。 这个差异是由于地壳岩石中的辐射散热(\textit{Rybach and Cermak, 1982})。

通常衰变放热率是从钾,铀和钍含量和岩石密度,利用公式(\textit{Rybach (1976) and Rybach and Cermak(1982)})计算出来的:

\begin{equation}
    \label{eq9}\\
    A=0.01\cdot \rho \cdot (9.52U+2.56Th+3.48K)
\end{equation}

A的单位是$\mu W/m^3$。$\rho$ 是岩石密度$g/cm^3$ U和Th是ppm为单位的浓度,K是以\%为单位的浓度。

在某些情况下,使用单位HGU(Heat Generation Unit):

\begin{align*}
    1\mu W/m^3=2.39HGU=2.39*10^{-13}cal/s/cm^3 \\
    1HGU=0.418*10^{-6}W/m^3=10^{-13}cal/s/cm^3
\end{align*}

在大多数火成岩中,铀和钍的贡献相当,而钾的贡献量总是小得多。占总产热的比例约为 40\% (U); 45\%(钍);和 15\% (K)。(\textit{Rybach and Cermak, 1982}) 表\ref{tb10}给了一些数据。

表\ref{tb11}展示了一下常见的沉积岩中钾、铀和钍含量和产热率。

\begin{table}[!h]
    \centering
    \caption{火成岩中衰变放热}
    \label{tb10}
    \begin{tabular}{l|l|l|l|l|lp{5cm}p{3cm}p{3cm}p{5cm}p{3cm}p{3cm}}
        \hline
        \multirow{2}*{侵入岩} & \multicolumn{2}{l}{$A(\mu W/m^3)$} & \multirow{2}*{喷出岩} & \multicolumn{2}{l}{$A(\mu W/m^3)$}\\
        ~ & 范围 & 均值 & ~ & 范围 & 均值 \\
        \hline
        花岗岩 & 0.7-7.65 & 3.00 & 安山岩 & \quad & 1.13 \\
        正长岩 & 1.1-5.9 & 2.84 & 玄武岩 & 0.2-0.95 & 0.63 \\
        闪长岩 & 0.2-2.45 & 1.15 & 流纹岩 & 1.9-4.0 & 3.58 \\
        辉长岩 & 0.1-0.73 & 0.33 & 英安岩 & 0.8-2.9 & 1.21 \\
        辉石岩 & 0.1-0.5 & 0.23 & 玢岩 & 0.7-1.7 & 0.94 \\
    \end{tabular}
\end{table}

\begin{table}[!h]
    \centering
    \caption{常见沉积岩中钾、铀和钍平均含量和产热率}
    \label{tb11}
    \begin{tabular}{l|l|l|l|l|lp{5cm}p{3cm}p{3cm}p{3cm}p{3cm}p{3cm}}
        \hline
        岩石类型 & 钾(\%) & 铀(ppm) & 钍(ppm) & 密度(g/cm) & $A(\mu W/m^3)$ \\
        \hline
        灰岩 & 0.3 & 2.0 & 1.5 & 2.6 & 0.62 \\
        白云岩 & 0.7 & 1.0 & 0.8 & 2.6 & 0.36 \\
        盐岩 & 0.1 & 0.02 & 0.01 & 2.2 & 0.012 \\
        硬石膏 & 0.4 & 0.1 & 0.3 & 2.9 & 0.090 \\
        页岩 & 2.7 & 3.7 & 12.0 & 2.4 & 1.8 \\
        黑色页岩 & 2.6 & 20.2 & 10.9 & 2.4 & 5.5 \\
        石英岩 & 0.9 & 0.6 & 1.8 & 2.4 & 0.32 \\
        长石砂岩 & 2.3 & 1.5 & 5.0 & 2.4 & 0.84 \\
        硬砂岩 & 1.3 & 2.0 & 7.0 & 2.4 & 0.99 \\
    \end{tabular}
\end{table}

衰变放热的测定一般需要光谱测量(实验室、测井或现场测量)。 在很多情况下,只有一个完整的伽马辐射记录可用。 \textit{Bucker and Rybach (1996)} 发表了一个从完整伽马辐射记录确定产生的热量的方法。 这种方法基于(完整)伽马辐射 (GR) 和产热A:

\begin{equation}
    \label{eq10}\\
    A=0.0158\cdot (GR-0.8)
\end{equation}

A的单位是$\mu W/m^3$,GR是伽马辐射强度,单位API。

作者指出,该方程适用于各种岩性从花岗岩到片麻岩、碳酸盐和角闪岩再到玄武岩。在 0–350 API 和 0.03–7 $\mu W/m^3$ 范围内,误差为小于 10\%。

\textit{Gegenhuber(2011a,b)} 开发了一种改进的方法,利用K、U 和 Th 的平均比率内容,来确定从伽马射线测井中产生的热量。

铀、钍和钾的含量随岩石类型而不同,并显示出从碱性到酸性火成岩,放射性热增加的趋势。这趋势也反映在与衰变放热相关的经验方程中以及密度和地震波速度,因为这两个参数都随着岩石酸性的增加而增加。示例在表\ref{tb12}。

\begin{table}[!h]
    \centering
    \caption{放热($\mu W/m^3$),地震波速(km/s)和密度($g/cm^3$)的经验性相关}
    \label{tb12}
    \resizebox{\textwidth}{15mm}{
    \begin{tabular}{l|l|lp{3cm}p{3cm}p{4cm}}
        \hline
        公式 & 岩石类型 & 参考来源 \\
        \hline
        $ln A=22.5-8.15\cdot \rho$ & 显生宙结晶岩石(瑞士) & Rybach and Buntebarth (1982) \\
        $ln A=22.54-8.145\cdot \rho$ & 花岗岩、玄武岩、辉长岩(贝加尔湖地区,俄罗斯) & Dorofeyeva (1990) \\
        $ln A=16.5-2.74\cdot V_p$ 50MPa& 显生宙结晶岩石(瑞士)& Rybach and Buntebarth (1982)\\
        $ln A=13.7-2.17\cdot V_p$ 100MPa& \quad & Cermak et al.(1990) \\
        $ln A=12.4-1.93\cdot V_p$ 200MPa& \quad & \quad  \\
    \end{tabular}}
\end{table}

\subsection{伽马辐射的相互作用}

基于伽马辐射与岩石相互作用进行探测的方法使用伽马源和伽马探测器。 结果取决于伽马辐射的能量和目标(岩石)的放射性特性。这会产生两个效应,对应两种测量结果模式:

\begin{enumerate}
    \item 光电效应,对应光电截面(PE)测量。
    \item 康普顿效应,对应伽马-伽马密度测量。
\end{enumerate}

\subsubsection{基础知识}

从伽马源发射的伽马光子与测量目标相互作用,会失去部分或全部能量。 在最简单的公式中,这中准直光束可以表示为:

\begin{equation}
    \label{eq11}\\
    \Psi = \Psi_0\cdot exp(-\alpha\cdot x)
\end{equation}

$\Psi$是穿过目标后的能流,$\Psi_0$是未穿过目标的能流,x是目标厚度,$\alpha$是目标吸收过程的系数。

伽马辐射与物质之间存在三个相互作用的过程:

\begin{enumerate}
    \item 光电效应(低能量)
    \item 康普顿效应(中等能量)
    \item 电子对生成(高能量)
\end{enumerate}

相互作用的概率不仅取决于伽马的能量源,也取决于目标的原子序数 Z。 图\ref{fig10}显示 Z与源三种效应的相对重要性能量图。灰色区域表示普通岩石的元素。

\begin{figure}[!h]
    \centering
    \includegraphics[width=15cm]{figures/fig10.jpg}
    \caption{三种伽马效应和Z的关系能量图。线表示两种效应发生概率一样。}
    \label{fig10}
\end{figure}

\paragraph{5.3.1.1 光电效应}

入射低能量的伽马光子 (<0.2 MeV) 与原子碰撞。 如果伽马光子的能量等于或超过离散结合能一个轨道电子,那么:

\begin{itemize}
    \item 伽马光子释放能量给环绕电子
    \item 电子离开其轨道并具有动能 $E_{kin}=\text{伽马射线能量}-\text{电子结合能}$。
\end{itemize}

光电效应的概率由伽马射线的能量和吸收物体的原子序数决定。在实际应用中,用到了两个参数:

\begin{itemize}
    \item 以 b/e 为单位的光电横截面指数(每个电子的Barn(一个面积单位))\\ 
    \begin{equation}
        \label{eq12}\\
        PE=(\frac{Z}{10})^{3.6}
    \end{equation}
    \item 体积光电截面 U,单位 $b/cm^3$:\\
    \begin{equation}
        \label{eq13}\\
        U=PE\cdot \rho_e
    \end{equation}
\end{itemize}

U 实际上与孔隙率无关,可以计算为组分的加权(按体积分数)平均值。

\paragraph{5.3.1.2 康普顿效应}

一个入射的中等能量伽马光子(伽马射线)与一个原子碰撞。它与一个轨道电子发生散射,并且仅传输其一部分能量。散射能量和射出电子的动能(康普顿或反冲电子)可以通过能量和动量守恒计算(\textit{Hearst and Nelson,1985})。 散射角依赖于能量而分布。

康普顿效应的概率由单位体积内的电子数决定。

\paragraph{5.3.1.3 电子对生成}

一个入射的高能伽马光子(伽马射线能量>1.022 MeV\footnote[1]{1.02MeV刚好是两倍电子静质量($mc^2$)})当它靠近原子核时可以转化为正负电子对。

光电效应和康普顿效应已经应用于地球科学。 两个效应都导致伽马辐射衰减,表示为在方程式\ref{eq11}中。吸收系数与不同效应有关。

\subsubsection{伽马-伽马-光电效应测量辨别矿物}


\end{document}